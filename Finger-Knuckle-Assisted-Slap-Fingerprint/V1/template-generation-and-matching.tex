\section{Template Generation and Matching\label{template-generation}}

\subsection{Finger Knuckle Template Generation\label{fk-template}}

Each of the segmented and normalized finger-knuckle images are subjected to the feature extraction to generate respective templates for the matching. There are a range of spatial domain \cite{sricharan2006knuckle}, \cite{kumar2009personal}, \cite{zhang2010online}, \cite{zhu2010multimodal}, \cite{zheng20163d}, \cite{kumar2016personal} and spectral domain \cite{aoyama2011finger}, \cite{kumar2015recovering} methods investigated in the literature to match finger knuckle patterns. Among these methods spatial-domain methods are quite attractive as they are computationally simpler and have shown to offer state-of-the-art results.  We employed the local feature descriptor based approach introduced in \cite{zheng20163d}, \cite{kumar2016personal} to generate finger knuckle templates as this approach is computationally attractive, accurate and generates smallest finger-knuckle templates (one-bit-per-pixel) that can be more conveniently stored along with the fingerprint-templates for the real applications. 

\subsection{Fingerprint Template Generation\label{fp-template}}

Among a range of methods introduced in the literature \cite{maltoni2009handbook} to match fingerprint images, minutiae-based methods are widely employed and therefore preferred. We considered a range of popular methods and implementations available in the references to generate fingerprint templates. Among various fingerprint matchers employed in the literature, NBIS (NIST Biometric Image Software) \cite{cappelli2010minutia} and minutiae cylinder code (MCC) \cite{watson2007user} are quite popular. We also considered a commercial off-the-shelf (COTS) matcher \cite{verifinger} which has shown quite accurate results in many references. Simultaneously acquired fingerprint images from the slap-fingerprint sensors are automatically segmented and employed to generate fingerprint templates. These templates are used to generate respective match scores that are consolidated for the user authentication. 

\subsection{Dynamic Score Consolidation\label{dynamic-score}}

The match scores generated from two independent pieces of evidences are consolidated to achieve a more reliable decision score for the user authentication. Biometrics literature \cite{maltoni2009handbook} provides extensive investigation on a range of methods to consolidate decisions from two pieces of evidences or features. Such consolidation from can be achieved at feature level, score level, or at the decision level. The score level combination is most widely used in the literature and is widely adapted \cite{jain2012biometric} in a range of biometrics system, largely due to its simplicity and the performance. The objective of our system is to serve as an add-on system on the top of existing or deployed slap-fingerprint system where the match scores are inherently available from the respective minutiae templates. Therefore, score level combination was preferred and adapted for the score consolidation in our system.  
\begin{algorithm}[h!]
    \renewcommand{\algorithmicrequire}{\textbf{Input:}}
    \renewcommand{\algorithmicensure}{\textbf{Output:}}
    \caption{Dynamic Match Score Consolidation}
    \begin{algorithmic}[1]
        \REQUIRE Match score $\bm{s_k, s_f, q_k, q_f}$ \\ 
        \ENSURE Consolidation score $\bm{s_c}$;\\
        \IF {$q_k \le 1$ and $q_f = 1$}
            \STATE $s_c = s_f$
        \ENDIF
        \IF {$q_k == 1$ and $q_f \le 1$}
            \STATE $s_c = s_k$
        \ENDIF
        \IF {$q_k \le 1$ and $q_f \le 1$}
            \STATE $s_c = 0$
        \ELSE
            \STATE $s_c = w \times s_k + (1-w)*s_f$
        \ENDIF
        \RETURN $s_c$
    \end{algorithmic}
    \label{algorithm-2}
\end{algorithm}

Real biometric systems during their deployments are often presented with missing or degraded quality of biometric samples and therefore the score-level consolidation scheme should be adaptive to such inputs. Therefore, we developed a dynamic scheme to consolidate match scores from two simultaneously generated match scores from the fingerprint and the finger-knuckle. Let the match score from fingerprint be represented by $s_f$ and the match score from the finger knuckle be represented by $s_k$. The confidence or the quality of input biometric image sample, as shown in Fig. \ref{block-diagram}, can be denoted as $q_k$ and $q_f$ respectively for the finger knuckle and fingerprint. The consolidated match score $s$ is generated as shown in Algorithm \ref{algorithm-2}. The weight $w ( 0 \leq w \leq 1)$  represents weight and is fixed empirically for all the experimental results in this paper. This scheme ensures that in case of missing or low quality biometric images, the importance is automatically granted for the other biometric modality.  

