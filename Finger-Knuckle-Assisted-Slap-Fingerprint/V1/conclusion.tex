\section{Conclusion and Further Work\label{conclusion}}

This paper introduces a new biometric system that can simultaneously acquire finger knuckle and fingerprint images, using the existing or widely deployed slap fingerprint devices, for more reliable user identification. We introduced a completely automated approach to reliably segment individual finger knuckle regions from the slap finger-print dorsal images with multiple fingers. More importantly, we developed a new database from such simultaneously acquired finger knuckle and fingerprint images using 4-4-2 protocols and provide \cite{datalink} the entire database to advance further research in this area. One of the key advantages of this system lies in its simplicity, as the finger-knuckle images are acquired using an additional or low-cost camera under the ambient illumination. Therefore, this system can be added to the existing or deployed slap-fingerprint devices at the border-crossings with the least inconvenience or the cost.    

The work detailed in this paper should be considered as the first attempt and further work is required to address the limitations. Firstly, our emphasis has been on simplicity and therefore a simplified dynamic fusion scheme was used to consolidate match scores from the simultaneously acquired biometric. More complex and dynamic fusion schemes that can also consider biometric image quality are expected to further improve the performance and are suggested for the further work. Secondly, the performance from the finger-knuckle detector needs improvement, especially for the thumb images. This can be attempted by using changing the detector backbone from mask RCNN to other detector \cite{redmon2016you} and enhancing the number/variety of training samples. Finally, this is first such attempt and therefore database from 120 volunteers could be acquired and none of these were paid or received any honorarium. A large-scale database from thousands of subjects is required to more effectively ascertain the performance improvement. In this work we only presented the performance improvement from the combination of two simultaneously acquired biometric for the individual fingers as the combination of all fingers can achieve full accuracy on relatively small user population. Therefore availability of large scale database can validate the effectiveness of performance improvement for a range of deployments applications which generally have millions of users for the identification.