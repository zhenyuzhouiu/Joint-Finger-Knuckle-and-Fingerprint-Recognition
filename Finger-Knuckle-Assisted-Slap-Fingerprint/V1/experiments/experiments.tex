\section{Experiments and Results}

\subsection{Database and Protocols}

There is no public database on the simultaneously acquired finger-knuckle and finger-print images from any slap-fingerprint sensor and this is the first system for the real applications in the best of our knowledge. Therefore, we developed an imaging setup to acquire the required database using an FBI compliant slap fingerprint sensor. We incorporated a commercial slap-fingerprint sensor [21] that also provides necessary drivers and SDK for the real applications. A 5M low-cost digital camera [29] was mounted on the top of slap fingerprint sensor and integrated to simultaneously acquire both biometric under single shot imaging. In order to acquire realistic finger-knuckle images for potential add-on solution with existing slap-fingerprint sensor deployments, so additional illumination was incorporated. Therefore, all the images were acquired under ambient illumination, in both indoor and outdoor environment.

The majority of the images in this database were acquired in India and none of the volunteers were paid for their contributions. All the volunteers aged between 11-62 and a large number of primary school students from a village in India contributed their images for this database. Each of the volunteers provided both right- and left-hand images in 4-4-2 protocol. We acquired 5 image samples from each of the subjects with each image sample acquired in 4-4-2 mode. Therefore, each subject provided a total of 30 images which included multiple fingerprint and finger knuckle images. A total of 120 subjects contributed to this database that included images from Indian, Chinese and European subjects. In order to advance further research and development efforts, the entire database acquired during this research is made publicly avail-able via [22]. Segmented images from each of the finger knuckle and fingerprints were used to generate respective templates as discussed in Section 3 and employed for the matching. We employed more challenging matching protocol to ensure more reliable estimation on the performance from the two biometric images. Therefore, each of the fingers generated 1200 ($120 \times 10$) genuine and 357000 ($120 \times 5 \times 5 \times 119$) impostor match scores, for each of the fingerprint and finger knuckle images, and were used to ascertain the performance.  